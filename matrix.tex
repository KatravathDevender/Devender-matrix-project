\documentclass{beamer}  
\usetheme{Warsaw}
\title{Introduction to AI and ML}
\subtitle{ Matrix Project}
\author{A.AVINASH, EE17BTECH11005 \and \\K.DEVENDER, EE17BTECH11015}
\begin{\newlinedocument}

\begin{frame}

\titlepage
 
\end{frame}  

\begin{frame}[t]{Question}

The point diametrically opposite to the point $P(1,0)$ on the circle $x^{2}+y^{2}+2x+4y-3=0$ is :

\vspace{1.5em}

Given circle in matrix form:
\vspace{1.5em}

\[x x^T+
x
\begin{bmatrix}
2 \\
4
\end{bmatrix}
=
3
\]

\end{frame}


\begin{frame}{Solution}
general equation of circle in matrix form:
\[x x^T 
-
2 x c^{T}
=
r^{2}
-
c c^{T}
\]
Here,
\newline
on comparing with our equation 
\newline
-2c^{T}
=\begin{bmatrix}
2\\
4

\end{bmatrix}
\newline
centre\hspace{2mm} of\hspace{2mm} circle 
\hspace{2mm}c
=
\begin{bmatrix}
-1 & -2

\end{bmatrix}
\newline
and
\newline
r^{2}-c c^{T}
=3
\newline
r^{2}-5=3
\newline
radius \hspace{2mm}of \hspace{2mm}circle
\hspace{2mm}r = 2^{3/2}
\end{frame}

\begin{frame}
Given $P(1,0)$ is the point on the circle 
\newline
\newline
Let $Y(a,b)$ be the diametrically opposite point to P.
\newline
\newline
As $Y(a,b)$ lies on circle and diametrically opposite to $P(1,0)$
\newline
\newline
So,
\newline
\newline
c is the mid point of  $P(1,0)$ and $Y(a,b)$ 

\end{frame}
   
\begin{frame}

c =\dfrac{P+Y}{2}
\newline
Y = 2c-P
\newline
Y = 2\begin{bmatrix}
-1 & -2\\
\end{bmatrix}
-
\begin{bmatrix}
1 & 0\\
\end{bmatrix}
\newline
Y=\begin{bmatrix}
-3 & -4\\
\end{bmatrix}
\newline
Therefore ,
\newline
Y=\begin{bmatrix}
-3 & -4\\
\end{bmatrix} \hspace{2mm} is\hspace{2mm}  the \hspace{2mm} diametrically\hspace{2mm}  opposite \hspace{2mm} point \hspace{2mm} to\hspace{2mm}  $P(1,0)$
\end{frame}
\end{document}